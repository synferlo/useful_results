
\documentclass[12pt]{article}
\usepackage{amsmath}
\usepackage{amssymb}
\usepackage{amsfonts}
\usepackage{mathrsfs}
\usepackage{bm}
\usepackage{indentfirst}
\setlength{\parindent}{0em}
\usepackage[margin=1in]{geometry}
\usepackage{graphicx}
\usepackage{setspace}
\doublespacing
\usepackage[flushleft]{threeparttable}
\usepackage{booktabs,caption}
\usepackage{float}
\usepackage{graphicx}

\usepackage{import}
\usepackage{xifthen}
\usepackage{pdfpages}
\usepackage{transparent}

\newcommand{\incfig}[1]{%
\def\svgwidth{\columnwidth}
\import{./figures/}{#1.pdf_tex}
}




\title{Useful Results in Econ}
\author{Synferlo}
\date{Dec. 15, 2020}


\begin{document}
\maketitle
\newpage


\section{SSOC for utility/profit maximization}

\subsection{Utility maximization}

Given $ U = U(X_1, X_2) $, quasi-concave, utility maximization requires
\begin{align}
				du  &= u_1dx_1 + u_2dx_2 = 0\\
				d^{2}u  &= (u_{11}dx_1 + u_{12}dx_2)dx_1 + (u_{21}dx_1 + u_{22}dx_2)dx_2
				< 0.
\end{align}
Therefore, we can rewrite equation (2)
\begin{align}
				&u_{11}dx_1^{2} + u_{12}dx_1dx_2 + u_{21}dx_1dx_2 + u_{22}dx_2^{2}  <0\\
				&u_{11}(dx_1^{2} + \frac{2u_{12}}{u_{11}}dx_1dx_2) + u_{22}dx_2^{2} <0\\
				&u_{11}\left(dx_1 + \frac{2u_{12}}{u_{11}}dx_1dx_2 + (\frac{u_{12}}{u_{11}}dx_2
				)^{2} - (\frac{u_{12}}{u_{11}}dx_2)^{2}\right) + u_{22}dx_2^{2} <0\\
				&u_{11}(dx_1 + \frac{u_{12}}{u_{11}}dx_2)^{2} + dx_2^{2}
				(u_{22} - \frac{u_{12}^{2}}{u_{11}}) < 0
\end{align}

where $ u_{11} < 0 $, $ (dx_1 + \frac{u_{12}}{u_{11}}dx_2)^{2} $ and $ dx_2^{2} $
are positive. Hence,
\begin{align}
				u_{22} - \frac{u_{12}^{2}}{u_{11}} < 0\\
				u_{11}u_{22} - u_{12}^{2} > 0.
\end{align}



\subsection{Profit maximization}

Given $ \pi = pf - w_1x_1 - w_2x_2 $, where $ f(\cdot ) $ stands for the 
production function, $ f = f(x_1, x_2) $.

\begin{align}
				d \pi  &= (pf_1 - w_1)dx_1 + (pf_2 - w_2)dx_2 =0\\
				d^{2} \pi  &= p(f_{11}dx_1 + f_{12}dx_2)dx_1 + p(f_{21}dx_1 + p_{22}dx_2)
				dx_2 <0.
\end{align}

Therefore, we can rewrite equation (10) as the following,
\begin{align}
				&p(f_{11}dx_1 + f_{12}dx_2)dx_1 + p(f_{21}dx_1 + f_{22}dx_2)dx_2 <0\\
				&f_{11}dx_1^{2} + f_{12}dx_1dx_2 + f_{12}dx_1dx_2 + f_{22}dx_2^{2} <0\\
				&f_{11}\left( dx_1 + \frac{f_{12}}{f_{11}}dx_2 \right) ^{2} + 
				\left( f_{22} - \frac{f_{12}^{2}}{f_{11}} \right) dx_2^{2} < 0.
\end{align}
Hence, we have similar results as in utility maximization problem,
\begin{align}
				&f_{22} - \frac{f_{12}^{2}}{f_{11}} <0\\
				&f_{11}f_{22} - f_{12}^{2} > 0.
\end{align}











\end{document}

